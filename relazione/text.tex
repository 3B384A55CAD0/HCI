\documentclass[paper=a4, fontsize=12pt]{scrartcl} % A4 paper and 11pt font size

% l'interlinea
\linespread{1.4}
\setlength{\parskip}{0.3cm plus4mm minus3mm} % spazio verticale paragrafi

\usepackage[italian]{babel} % Italian language
\usepackage[utf8x]{inputenc} % utf8

\usepackage[T1]{fontenc} % Use 8-bit encoding that has 256 glyphs
\usepackage{fourier} % Use the Adobe Utopia font

\usepackage{graphicx}

\usepackage{amsmath,amsfonts,amsthm} % Math packages

\usepackage{sectsty} % Allows customizing section commands
\allsectionsfont{ \normalfont\scshape} % Make all sections centered, the default font and small caps

\usepackage{fancyhdr} % Custom headers and footers
\pagestyle{fancyplain} % Makes all pages in the document conform to the custom headers and footers
\fancyhead{} % No page header - if you want one, create it in the same way as the footers below
\fancyfoot[L]{} % Empty left footer
\fancyfoot[C]{} % Empty center footer
\fancyfoot[R]{\thepage} % Page numbering for right footer
\renewcommand{\headrulewidth}{0pt} % Remove header underlines
\renewcommand{\footrulewidth}{0pt} % Remove footer underlines
\setlength{\headheight}{13.6pt} % Customize the height of the header

\usepackage[font=small,labelfont=bf]{caption} %Style of the caption of figures
\usepackage{float} %to force the figure position

\setlength\parindent{0pt} % Removes all indentation from paragraphs

% Style dell'abstract
\usepackage{abstract}

%----------------------------------------------------------------------------------------
% TITLE
%----------------------------------------------------------------------------------------

\newcommand{\horrule}[1]{\rule{\linewidth}{#1}}

\title{	
\normalfont \normalsize 
\textsc{Università degli studi di Bologna} \\ [25pt]
\horrule{0.5pt} \\[0.4cm]
\huge Titolo \\
\horrule{2pt} \\[0.5cm]
}

\author{Autore}

\date{\normalsize\today}

\begin{document}

\maketitle

\begin{abstract}
Lo scopo del progetto Visual Directions consiste nel semplificare la memorizzazione preventiva di percorsi (principalmente cittadini). Lo stesso è rivolto a chiunque intenda studiare le caratteristiche fondamentali di un percorso tra due punti (non eccessivamente distanti l’uno dall’altro) al fine di poterlo poi percorrere senza l’ausilio di mappe, navigatori o altri mezzi di orientamento. Attraverso una accurata fase di design ed una intensiva fase di testing è stato possibile misurare la resa del progetto in termini di efficienza, efficacia e soddisfazione.
\end{abstract}

%----------------------------------------------------------------------------------------
% INTRODUZIONE
%----------------------------------------------------------------------------------------

\section{Introduzione}

Mia noonna


%Bibliografia%---------------------------------------------------------------
\bibliographystyle{unsrt}
\addcontentsline{toc}{chapter}{\refname}\nocite{*}
\bibliography{text}
%----------------------------------------------------------------------------

\end{document}